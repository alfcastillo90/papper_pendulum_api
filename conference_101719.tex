\documentclass[conference]{IEEEtran}
\IEEEoverridecommandlockouts
% The preceding line is only needed to identify funding in the first footnote. If that is unneeded, please comment it out.
\usepackage{cite}
\usepackage{amsmath,amssymb,amsfonts}
\usepackage{algorithmic}
\usepackage{graphicx}
\usepackage{textcomp}
\usepackage{xcolor}
\def\BibTeX{{\rm B\kern-.05em{\sc i\kern-.025em b}\kern-.08em
    T\kern-.1667em\lower.7ex\hbox{E}\kern-.125emX}}
\begin{document}

\title{Optimización en la asignación de vacunas en centros de salud mediante una API RESTful basada en el algoritmo de búsqueda de Péndulo\\
{\footnotesize \textsuperscript{*}Note: Sub-titles are not captured in Xplore and
should not be used}
\thanks{Identify applicable funding agency here. If none, delete this.}
}

\author{\IEEEauthorblockN{Carlos Alfredo Castillo Rodriguez}
    \IEEEauthorblockA{\textit{Facultad de Ingeniería, ciencia y tecnología} \\
        \textit{Universidad Bernardo O'higgins}\\
        Santiago, Chile \\
        castilloc@postgrado.ubo.cl}}

\maketitle


\renewcommand{\abstractname}{Resumen}
\begin{abstract}
    \vspace{\baselineskip}

    En los últimos años, existen evidencias de desabastecimiento de medicamentos en países subdesarrollados y en vías de desarrollo, por razones principalmente políticas o climáticas. (P2) Dicho desabastecimiento genera consecuencias negativas en la salud de la población. En especial en América Latina, la falta de vacunas ha sido la principal dificultad para ampliar la cobertura de vacunación, debido a la alta dependencia de las importaciones de medicamentos y materias primas para el desarrollo de tecnologías. Lograr minimizar la falta de medicamentos es esencial para garantizar el derecho a la salud de las personas y fortalecer los programas de prevención y control de enfermedades.\par

    Cuando se habla de problemas de optimización, a menudo implican minimizar o maximizar recursos tales como pérdidas o ganancias. Estos problemas son usualmente complejos y una solución con algoritmos de optimización exactos es de alto costo computacional.\par

    Este trabajo tiene como objetivo proponer una solución para optimizar la distribución de medicamentos en centros de salud utilizando un algoritmo de Búsqueda del Péndulo (PSA) implementado en una API Restful con Node.js y MongoDB. La solución propuesta se enfoca en la utilización de la metaheurística como método de optimización y la implementación de una aplicación web que permita a los usuarios realizar la optimización de manera eficiente y efectiva. Se basó en una revisión bibliográfica exhaustiva de los algoritmos metaheurísticos y su aplicación en la optimización de la distribución de medicamentos en centros de salud. Se estudió en profundidad el algoritmo de Búsqueda del Péndulo y se propuso su implementación en una API Restful utilizando Node.js y MongoDB para la optimización de la distribución de medicamentos.\par

    La propuesta de solución presentada podría ser una alternativa viable y efectiva para resolver el problema de la distribución de medicamentos en centros de salud. La utilización de la metaheurística como método de optimización permitiría encontrar soluciones óptimas en tiempos razonables y con costos computacionales acotados. (P6) Además, la implementación de una API Restful utilizando Node.js y MongoDB permitiría una fácil integración con otras aplicaciones y sistemas existentes.\par

    En resumen, se propone una solución para optimizar la distribución de medicamentos en centros de salud utilizando un algoritmo de Búsqueda del Péndulo implementado en una API Restful con Node.js y MongoDB. La solución propuesta se enfoca en la utilización de la metaheurística como método de optimización y la implementación de una aplicación web eficiente y efectiva.
\end{abstract}




\begin{IEEEkeywords}
    component, formatting, style, styling, insert
\end{IEEEkeywords}

\section{Introducción}

Este trabajo se centra en abordar la creciente crisis de la escasez de medicamentos esenciales en todo el mundo, la cual ha llevado a malos resultados sanitarios y al uso inapropiado de medicamentos. El suministro insuficiente en sistemas de salud, evidente especialmente en América Latina y el Caribe, impide satisfacer las necesidades de salud pública y de los pacientes. \cite{OMS2016}.

Nuestro objetivo es implementar una solución de software basada en algoritmos metaheurísticos, particularmente el algoritmo de Búsqueda del Péndulo, para optimizar la distribución de medicamentos en los centros de salud. La propuesta es desarrollar una API RESTful utilizando Node.js y MongoDB que implemente este algoritmo, con el fin de proporcionar una solución que busque la distribución óptima en tiempos razonables y con costos computacionales acotados.

En el ámbito de la distribución de medicamentos, la optimización juega un papel fundamental para mejorar la eficiencia y disponibilidad en los centros de salud. Para abordar este problema, se han utilizado modelos matemáticos y algoritmos computacionales, entre ellos los algoritmos metaheurísticos, que proporcionan soluciones aproximadas a problemas complejos de optimización combinatoria.

En este contexto, este trabajo tiene como objetivo desarrollar una solución de software basada en metaheurísticas para optimizar la distribución de medicamentos en centros de salud. Se propone la implementación del algoritmo de Búsqueda del Péndulo (PSA) en una API RESTful utilizando Node.js y MongoDB. Además, se plantea el desarrollo de una aplicación web que utilice esta API para que los usuarios puedan realizar la optimización de la distribución de medicamentos de manera eficiente y efectiva, considerando la factibilidad y viabilidad del modelo propuesto en la situación actual de Chile.

Los objetivos específicos de este trabajo son los siguientes:
\begin{enumerate}
    \item Estudiar a fondo el algoritmo de Búsqueda del Péndulo y entender su implementación y parámetros necesarios.
    \item Implementar una API RESTful que incluya la implementación del algoritmo para la optimización de la distribución de medicamentos en centros de salud.
    \item Realizar pruebas y evaluaciones del algoritmo en la API en diferentes escenarios de distribución de medicamentos para validar su eficacia y eficiencia.
    \item Generar una interfaz visual con Swagger para facilitar el uso de la API por parte de los usuarios.
    \item Evaluar la factibilidad y viabilidad de la aplicación web desarrollada, teniendo en cuenta aspectos como la escalabilidad, seguridad, usabilidad y costos.
\end{enumerate}

\section{Marco teórico}
\label{sec:MT}

\subsection{Metaheurística}
Las metaheurísticas son técnicas de optimización utilizadas para resolver problemas complejos, proporcionando soluciones aceptables en un tiempo más corto en comparación con los algoritmos de búsqueda exacta. Son altamente adaptables y no imponen ninguna exigencia sobre la formulación del problema de optimización \cite{soerensen2010}. Diversos mecanismos y estrategias, como un peso de inercia decreciente con el tiempo y operaciones de cruce, son adoptados para mejorar la eficiencia de la búsqueda \cite{eberhart2000, bansal2011, elkhateeb2013, yang2011, sarangi2016, binkley2008}.

\subsection{Node.js}
Node.js es un entorno de ejecución de código abierto que permite la programación del lado del servidor utilizando JavaScript. Proporciona un modelo de programación basado en eventos y un sistema de gestión de paquetes integrado, facilitando el desarrollo de aplicaciones escalables y de alta concurrencia \cite{shah2017}.

\subsection{MongoDB}
MongoDB es un sistema de gestión de bases de datos NoSQL orientado a documentos que proporciona un alto rendimiento de lectura y escritura. Utiliza un formato similar a JSON llamado BSON, que se adapta naturalmente a las metodologías de programación orientadas a objetos. Soporta consultas complejas y ofrece características como fragmentación automática y replicación \cite{krishnan2016}.

\subsection{JSON Web Tokens (JWT)}
Los JSON Web Tokens son un mecanismo de autenticación y autorización que codifica un conjunto de afirmaciones como un objeto JSON en una estructura. Se utilizan en aplicaciones web para mantener a los usuarios autenticados y autorizados durante su sesión \cite{IETF2015}.

\subsection{RESTful API}
Las APIs RESTful son interfaces de programación basadas en HTTP que permiten la creación, lectura, actualización y eliminación de recursos. Se utilizan en el diseño de microservicios y admiten la interoperabilidad y la WWW. Los principios de REST, como la interfaz uniforme y la identificación de recursos, permiten una arquitectura simple y escalable \cite{ehsan2022}.


\section{Resultados}
\label{sec:Res}
\subsection{Resultados 1}

\subsection{Resultados 2}

\begin{itemize}
    \item T1.
    \item T2.
    \item T3.
\end{itemize}

Tabla~\ref{tab:1}
Fig.~\ref{fig:1}

\begin{table}[htbp]
    \caption{Table Type Styles}
    \begin{center}
        \begin{tabular}{|c|c|c|c|}
            \hline
            \textbf{Table} & \multicolumn{3}{|c|}{\textbf{Table Column Head}}                                                         \\
            \cline{2-4}
            \textbf{Head}  & \textbf{\textit{Table column subhead}}           & \textbf{\textit{Subhead}} & \textbf{\textit{Subhead}} \\
            \hline
            copy           & More table copy$^{\mathrm{a}}$                   &                           &                           \\
            \hline
            \multicolumn{4}{l}{$^{\mathrm{a}}$Sample of a Table footnote.}
        \end{tabular}
        \label{tab:1}
    \end{center}
\end{table}

\begin{figure}[htbp]
    \centerline{\includegraphics{Figures/fig1.png}}
    \caption{Figura Ejemnplo.}
    \label{fig:1}
\end{figure}


\section{Trabajos Relacionados}
\label{sec:TR}
...


\section{Conclusiones}
\label{sec:Conclusiones}
...

\bibliographystyle{IEEEtran}
\bibliography{EjemploBIO}

\end{document}
