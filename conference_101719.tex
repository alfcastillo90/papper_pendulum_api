\documentclass[conference]{IEEEtran}
\IEEEoverridecommandlockouts
% The preceding line is only needed to identify funding in the first footnote. If that is unneeded, please comment it out.
\usepackage{cite}
\usepackage{amsmath,amssymb,amsfonts}
\usepackage{algorithmic}
\usepackage{graphicx}
\usepackage{textcomp}
\usepackage{xcolor}
\def\BibTeX{{\rm B\kern-.05em{\sc i\kern-.025em b}\kern-.08em
    T\kern-.1667em\lower.7ex\hbox{E}\kern-.125emX}}
\begin{document}

\title{Optimización en la asignación de vacunas en centros de salud mediante una API RESTful basada en el algoritmo de búsqueda de Péndulo\\
{\footnotesize \textsuperscript{*}Note: Sub-titles are not captured in Xplore and
should not be used}
\thanks{Identify applicable funding agency here. If none, delete this.}
}

\author{\IEEEauthorblockN{1\textsuperscript{st} Given Name Surname}
\IEEEauthorblockA{\textit{dept. name of organization (of Aff.)} \\
\textit{name of organization (of Aff.)}\\
City, Country \\
email address or ORCID}
\and
\IEEEauthorblockN{2\textsuperscript{nd} Given Name Surname}
\IEEEauthorblockA{\textit{dept. name of organization (of Aff.)} \\
\textit{name of organization (of Aff.)}\\
City, Country \\
email address or ORCID}}

\maketitle


\renewcommand{\abstractname}{Resumen}
\begin{abstract}
\vspace{\baselineskip}

En los últimos años, existen evidencias de desabastecimiento de medicamentos en países subdesarrollados y en vías de desarrollo, por razones principalmente políticas o climáticas. (P2) Dicho desabastecimiento genera consecuencias negativas en la salud de la población. En especial en América Latina, la falta de vacunas ha sido la principal dificultad para ampliar la cobertura de vacunación, debido a la alta dependencia de las importaciones de medicamentos y materias primas para el desarrollo de tecnologías. Lograr minimizar la falta de medicamentos es esencial para garantizar el derecho a la salud de las personas y fortalecer los programas de prevención y control de enfermedades.\par
    
Cuando se habla de problemas de optimización, a menudo implican minimizar o maximizar recursos tales como pérdidas o ganancias. Estos problemas son usualmente complejos y una solución con algoritmos de optimización exactos es de alto costo computacional.\par
    
Este trabajo tiene como objetivo proponer una solución para optimizar la distribución de medicamentos en centros de salud utilizando un algoritmo de Búsqueda del Péndulo (PSA) implementado en una API Restful con Node.js y MongoDB. La solución propuesta se enfoca en la utilización de la metaheurística como método de optimización y la implementación de una aplicación web que permita a los usuarios realizar la optimización de manera eficiente y efectiva. Se basó en una revisión bibliográfica exhaustiva de los algoritmos metaheurísticos y su aplicación en la optimización de la distribución de medicamentos en centros de salud. Se estudió en profundidad el algoritmo de Búsqueda del Péndulo y se propuso su implementación en una API Restful utilizando Node.js y MongoDB para la optimización de la distribución de medicamentos.\par
    
La propuesta de solución presentada podría ser una alternativa viable y efectiva para resolver el problema de la distribución de medicamentos en centros de salud. La utilización de la metaheurística como método de optimización permitiría encontrar soluciones óptimas en tiempos razonables y con costos computacionales acotados. (P6) Además, la implementación de una API Restful utilizando Node.js y MongoDB permitiría una fácil integración con otras aplicaciones y sistemas existentes.\par
    
En resumen, se propone una solución para optimizar la distribución de medicamentos en centros de salud utilizando un algoritmo de Búsqueda del Péndulo implementado en una API Restful con Node.js y MongoDB. La solución propuesta se enfoca en la utilización de la metaheurística como método de optimización y la implementación de una aplicación web eficiente y efectiva.
\end{abstract}




\begin{IEEEkeywords}
component, formatting, style, styling, insert
\end{IEEEkeywords}

\section{Introducción}
....
Según  \cite{Hebert2022}, asdasdads
.

Este artículo se estructura de la siguiente forma. Sección~\ref{sec:MT} presenta las bases de.... Secciób~\ref{sec:Res} destaca los principales resultados de este trabajo. Sección~\ref{sec:TR} describe trabajos relacionados con el foco de investigación de estre trabajo. Finalmente, Sección~\ref{sec:Conclusiones} entrega las ideas finales y trabajos futuras de este trabajo de investigación.

\section{Marco teórico} 
\label{sec:MT}
\subsection{Tema 1}

\subsection{Tema 2}

\subsection{Tema 3}

\section{Resultados}
\label{sec:Res}
\subsection{Resultados 1}

\subsection{Resultados 2}

\begin{itemize}
\item T1.
\item T2.
\item T3.
\end{itemize}

Tabla~\ref{tab:1}
Fig.~\ref{fig:1}

\begin{table}[htbp]
\caption{Table Type Styles}
\begin{center}
\begin{tabular}{|c|c|c|c|}
\hline
            \textbf{Table} & \multicolumn{3}{|c|}{\textbf{Table Column Head}}                                                         \\
\cline{2-4} 
            \textbf{Head}  & \textbf{\textit{Table column subhead}}           & \textbf{\textit{Subhead}} & \textbf{\textit{Subhead}} \\
\hline
            copy           & More table copy$^{\mathrm{a}}$                   &                           &                           \\
\hline
\multicolumn{4}{l}{$^{\mathrm{a}}$Sample of a Table footnote.}
\end{tabular}
\label{tab:1}
\end{center}
\end{table}

\begin{figure}[htbp]
\centerline{\includegraphics{Figures/fig1.png}}
\caption{Figura Ejemnplo.}
\label{fig:1}
\end{figure}


\section{Trabajos Relacionados}
\label{sec:TR}
...


\section{Conclusiones}
\label{sec:Conclusiones}
...

\bibliographystyle{IEEEtran}  
\bibliography{EjemploBIO}

\end{document}
